\newpage
\section {Билет 7. Распределенные запросы. Распределенные и кластерные базы данных. Распределенные транзакции.}

1) Про распределённые транзакции.

Транзакция - это набор связанных задач, который, помимо всего прочего, завершается успешно (фиксация) или с ошибкой (отмена) как единое целое. \href{https://docs.microsoft.com/ru-ru/dotnet/framework/data/adonet/distributed-transactions}{Распределенная транзакция} — это транзакция, затрагивающая несколько ресурсов. Для фиксации распределенной транзакции все участники должны гарантировать, что любое изменение данных будет постоянным. Изменения должны сохраняться даже в случае фатального сбоя системы или других непредвиденных событий. Если хоть один из участников не сможет предоставить такую гарантию, вся транзакция завершится с ошибкой и будет выполнен откат любых изменений данных внутри области транзакции.

\bigskip
2) Про распределённые базы данных (про кластерные лучше просто прочесть \href{https://www.jetinfo.ru/klasternye/#gl_2_1}{здесь}).

\href{https://ru.wikipedia.org/wiki/%D0%A0%D0%B0%D1%81%D0%BF%D1%80%D0%B5%D0%B4%D0%B5%D0%BB%D1%91%D0%BD%D0%BD%D0%B0%D1%8F_%D0%B1%D0%B0%D0%B7%D0%B0_%D0%B4%D0%B0%D0%BD%D0%BD%D1%8B%D1%85}{Распределённая БД} - это БД, составные части которой размещаются в различных узлах компьютерной сети в соответствии с каким-либо критерием.

Распределённая база данных — это именно единая база данных, а не произвольный набор файлов, индивидуально хранимых на разных узлах сети и являющейся распределенной файловой системой. Данные представляют собой РБД, только если они связаны в соответствии с некоторым структурным формализмом, реляционной моделью, а доступ к ним обеспечивается единым высокоуровневым интерфейсом.

Распределённые базы могут иметь разный уровень реплицированности — от полного отсутствия дублирования информации, до полного дублирования всей информации во всех распределённых копиях (например, блокчейн).

Распределение (включая фрагментацию и репликацию) базы данных по множеству узлов невидимо для пользователей. Это свойство называется прозрачностью, а технология распределения и реплицирования данных по множеству компьютеров, связанных сетью, является основополагающей для реализации концепции независимости данных от среды хранения. Это обеспечивается за счёт нескольких видов прозрачности:

\begin{enumerate}
	\item[\textbullet] прозрачность сети, а следовательно, прозрачность распределения
	\item[\textbullet] прозрачность репликации
	\item[\textbullet] прозрачность фрагментации
	\item[\textbullet] прозрачность доступа, означающая, что пользователи имеют дело с единым логическим образом базы данных и осуществляют доступ к распределенным данным точно так же, как если бы они хранились централизованно.
\end{enumerate}

В идеале полная прозрачность подразумевает наличие языка запросов к распределённой СУБД, не отличающегося от языка для централизованной СУБД.

\bigskip
3) Про распределённые запросы лучше просто почитать \href{https://www.ibm.com/docs/ru/informix-servers/12.10?topic=SSGU8G_12.1.0/com.ibm.admin.doc/ids_admin_0069.htm}{здесь} и \href{https://studfile.net/preview/9032573/page:10/}{вот здесь}.
