\newpage
\section {Билет 12. Полнотекстовый поиск. Алгоритмы без индексов. Поиск с реверсивным индексом.}

Текстом назовем произвольный упорядоченный набор слов в некотором алфавите. Фрагмент текста или просто фрагмент — поднабор текста, в который входят подряд идущие слова с сохранением порядка.
\href{https://ru.wikipedia.org/wiki/Полнотекстовый_поиск}{Полнотекстовый поиск} - автоматизированный поиск документов, при котором поиск ведётся не по именам документов, а по их содержимому, всему или существенной части.

\subsection{Безиндексный поиск}
\href{https://habr.com/ru/company/vk/blog/270507/}{regex}

\subsection {Реверсивный индекс}
Пусть у нас выборка текстов $T_1, T_2, \dots T_{N}$, занумерованные натуральными числами. Есть несколько ключевых слов $k_1, k_2, \dots k_n$ и есть файлы $F_1, \dots F_n$. В файле $F_i$ записаны номера текстов, в которых встречается слово $k_i$. Приведем пример, пусть есть два слова СТОЛ, СТУЛ, и файлы $F_1, F_2$  имеют вид: \\
$$ F_1: 1,3,4,7,9 $$
$$ F_2: 2,3,4,5,6,8 $$

Пусть номера текстов в файлах отсортированы!
Тогда алгоритм поиска прост: мы производим сливание отсортированных массивов (соответственно, если нам нужно найти слова СТОЛ И СТУЛ, то выбираем файлы в которые входят оба эти слова, если  СТОЛ ИЛИ СТУЛ, то просто сливаем два массива). Так как массивы в файлах отсортированы, то задача слияния выполняется быстро.\\
Вопрос: если один файл изменили, что делать ? Например в 1 тексте слово ТАБУРЕТКА заменили на СТУЛ. Текст 1 попадает в специальный файл где хранятся удаленные тексты, а номер 1 заменяется, например, на 10. Тогда:
$$ F_1: 3,4,7,9, 10 $$
$$ F_2: 2,3,4,5,6,8,10 $$
$$DEL (\text{удаленные}): 1$$

Если в 4 файле заменили, слово СТУЛ на ТАБУРЕТКА, то (номер 4 заменяем на 11) имеем:
$$ F_1: 3,7,9, 10, 11 $$
$$ F_2: 2,3,5,6,8,10 $$
$$DEL (\text{удаленные}): 1, 4$$

Когда все номера становятся очень большими их все можно уменьшить.