\documentclass[specialist, subf, href, colorlinks=true, 14pt, times, mtpro, final]{report}

\usepackage[utf8x]{inputenc}
\usepackage[english, russian]{babel}
\usepackage[T2A]{fontenc}
\usepackage{amsmath,amsthm,amssymb}
\usepackage {wrapfig}
\usepackage {enumitem}  
\usepackage{graphicx}
\usepackage{multicol}
\usepackage{mathrsfs}
\usepackage{xcolor}
\usepackage{hyperref}
\usepackage{tikz}
\usepackage{pdfpages}


\usetikzlibrary{decorations.pathreplacing}
\usepackage[a4paper, mag=1000, includefoot, left=1.5cm, right=1.5cm, top=1cm, bottom=1cm, headsep=1cm, footskip=1cm]{geometry}
\usepackage{floatrow}
\usepackage{tikz}
\newcommand{\RNumb}[1]{\uppercase\expandafter{\romannumeral #1\relax}}
\usetikzlibrary{graphs}

\theoremstyle{definition}
\newtheorem{defn}{Определение}[section]
\newtheorem{example}{Пример}[section]
\newtheorem{state}{Утверждение}[section]
\newtheorem{theorem}{Теорема}[section]
\newtheorem{lemma}{Лемма}[section]
\newtheorem{axiom}{Аксиома}[section]
\newtheorem{consequence}{Следствие}[section]

\begin{document}
	\centering
	{\bf Программа спецкурса <<Введение в распределенные системы>>}
	\begin{enumerate}
\item Реляционная база данных. Область применимости и основные понятия. Отношение, атрибут, кортеж. Связь. Целостность данных.
\item Операции над данными. Объединение, пересечение и разность. Произведение и соединение по условию. Деление. Ограничение, проекция и переименование. Реляционное исчисление.
\item Нормальные формы.
\item Индексы (дерево, карты, хэш).
\item Оптимизация запросов, построение и оценка планов запросов.
\item Транзакции, блокировки, версионность.
\item Распределенные запросы. Распределенные и кластерные базы данных. Распределенные транзакции.
\item Одно-, двух- и трехзвенные архитектуры.
\item IP, TCP, UDP, VPN, NAT
\item Методы обеспечения отказоустойчивости. RAID, распределенное хранение данных, виды и репликаций.
\item Метода шифрования данных.
\item Полнотекстовый поиск. Алгоритмы без индексов. Поиск с реверсивным индексом.
\item Обработка текстов. Морфологический анализ, tf*idf, ранжирование, исправление опечаток, классификация, кластеризация, поиск дубликатов.

\item Свойства распределенных систем: конкурентность, отсутствие состояния, независимые сбои. Основные проблемы и аспекты проектирования: неоднородность, открытость, безопасность, масштабируемость, отказоустойчивость, конкурентность, прозрачность.
\item Модели взаимодействия, ошибок, безопасности.
\item Архитектуры информационных систем: "монолитная", клиент-серверная, многоуровневая. Одноранговые системы (peer-to-peer).
\item Основные уровни сетевых протоколов. Семиуровневая модель OSI. Сети пакетной коммутации. Маршрутизация сообщений.
\item Невозможность гарантированной доставки сообщений. Алгоритм скользящего окна.
\item Многоадресная передача. Протоколы B-, R-, CO-, ТО-multicast: устойчивость к сбоям, сложность по числу сообщений и времени.
\item Отношение причинно-следственной зависимости. Метки Лэмпорта, векторные часы.
\item Алгоритмы избрания лидера. Свойства завершения, единственности и согласия. Выборы в кольцевой сети. Выборы на основе протоколов многоадресной передачи. Справедливость.
\item Взаимное исключение. Решение на основе сервера. Алгоритм Дейкстры с общими регистрами. Алгоритм Петерсона для двух процессов. Алгоритм на основе голосования (Maekawa).
\item Согласие в распределенной системе. Задача византийских генералов: распространение значения (agreement), согласование решения (consensus), согласование вектора (interactive consistency). Невозможность решения при трёх процессах и одном сбое. Алгоритм Лэмпорта для "устных" сообщений. Пример распределенных транзакций.
\item Отказоустояивость. Активная и пассивная репликация. Алгоритмы поддержания согласованного состояния реплик. Построение надежного хранилища из ненадежных компонентов.
\item Модели логического разграничения доступа: мандатная, дискреционная, ролевая. Атрибутивная модель.
\item Конфиденциальные вычисления. Гомоморфное шифрование и многосторонние вычисления.
	\end{enumerate}


\end{document}
