\newpage
\section{Билет 2. Операции над данными. Объединение, пересечение и разность. Произведение и соединение по условию. Деление. Ограничение, проекция и переименование. Реляционное исчисление.}

Два отношения называются  \textbf{совместимыми по типу}, если каждое из них имеет одинаковое множество имен атрибутов и если соответствующие атрибуты определены на одном и том же домене. 
(Домен атрибута — множество допустимых значений, которые может принимать атрибут). 

\textit{Замечание:} никакие реляционные операторы не передают результатирующему отношению никаких данных о потенциальных ключах.

\begin{enumerate}
    \item   \textbf{Объединением} двух совместимых по типу отношений A и B называется отношение с тем же заголовком, что и у отношений A и B, и телом, состоящим из кортежей, принадлежащих или A, или B, или обоим отношениям.

Синтаксис операции объединения: A UNION B


\textit{Замечание1}: Объединение, как и любое отношение, не может содержать одинаковых кортежей. Поэтому, если некоторый кортеж входит и в отношение A, и отношение B, то в объединение он входит один раз.


\item 
\textbf{Пересечением} двух совместимых по типу отношений A и B называется отношение с тем же заголовком, что и у отношений A и B , и телом, состоящим из кортежей, принадлежащих одновременно обоим отношениям A и B.

Синтаксис операции пересечения: A INTERSECT B

\item  \textbf{Вычитанием (или разностью)} двух совместимых по типу отношений A и B называется отношение с тем же заголовком, что и у отношений A и B, и телом, состоящим из кортежей, принадлежащих отношению A и не принадлежащих отношению B. 

Синтаксис операции вычитания: A MINUS B

\item \textbf{Декартовым произведением} двух отношений $A(A_1, ..., A_n)$ и $B(B_1,..., B_m)$ называется отношение, заголовок которого является сцеплением заголовков отношений A и B: $(A_1, ..., A_n, B_1,..., B_m)$,
а тело состоит из кортежей, являющихся сцеплением кортежей отношений A и B: $(a_1, ..., a_n, b_1,..., b_m)$, таких, что , $(a_1, ..., a_n) \in A, (b_1,..., b_m) \in B$

Синтаксис операции произведения: A TIMES В

\item    \textbf{Соединением} отношений A и B по условию c называется отношение (A TIMES B) WHERE c, где c представляет собой логическое выражение, в которое могут входить атрибуты отношений А и B и (или) скалярные выражения.

Таким образом, операция соединения есть результат последовательного применения операций декартового произведения и выборки. 


\item Пусть даны отношения $A(X_1, .., X_n, Y_1, ..., Y_m)$  $B(Y_1, ..., Y_m)$, причем атрибуты $Y_1, ..., Y_m$ - общие для двух отношений. \textbf{Делением} отношений A на B называется отношение с заголовком  $X_1, .., X_n$ и телом, содержащим множество кортежей $(x_1, .., x_n)$ , таких, что для всех кортежей $(y_1, .., y_m) \in B$, в отношении A найдется кортеж $(x_1, .., x_n, y_1, .., y_m)$.

Отношение A выступает в роли делимого, отношение B выступает в роли делителя. Деление отношений аналогично делению чисел с остатком.

Синтаксис операции деления: A DEVIDBY B


\item  \textbf{Выборка (или ограничение)} возвращает отношение, содержащее кортежи из заданного отношения, которые удовлетворяют указанным условиям;

Синтаксис операции выборки: A WHERE c, c - условие 

\item \textbf{Проекцией} отношения A по атрибутам X, Y, ..., Z, где каждый из атрибутов принадлежит отношению A, называется отношение с заголовком (X, Y, ..., Z)  и телом, содержащим множество кортежей вида (x, y, ..., z), таких, для которых в отношении A найдутся кортежи со значением атрибута X равным x, значением атрибута Y равным y, …, значением атрибута Z равным z. 

Синтаксис операции проекции: A[X, Y, .., Z]

\item Операция \textbf{переименования} производит отношение, тело которого совпадает с телом операнда, но имена атрибутов изменены. 

Cинтаксис: R RENAME Atr1, Atr1, ... AS NewAtr1, NewAtr2, ..; где R - отношение, Atr1, Atr2.. - исходные имена атрибутов, NewAtr1, NewAtr2.. - новые имена атрибутов.

\end{enumerate}

\textbf{Реляционное исчисление} — декларативный язык для работы с отношениями, описывающий какими свойствами должен обладать требуемый результат.
