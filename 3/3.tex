\newpage
\section {Билет 3. Нормальные формы.}

Возможно полезно почитать и посмотреть примеры тут:

\url{https://habr.com/ru/post/254773/} 

\url{https://info-comp.ru/database-normalization}

\textbf{Нормальная форма} — требование, предъявляемое к структуре таблиц в теории реляционных баз данных для устранения из базы избыточных функциональных зависимостей между атрибутами (полями таблиц).

Цель нормализации: исключить избыточное дублирование данных, которое является причиной аномалий, возникших при добавлении, редактировании и удалении кортежей(строк таблицы).

3.1 \textbf{Первая нормальная форма}

Чтобы база данных находилась в 1 нормальной форме, необходимо чтобы ее таблицы соблюдали следующие реляционные принципы:

\begin{itemize}
	\item В таблице не должно быть дублирующих строк
	\item В каждой ячейке таблицы хранится атомарное значение (одно не составное значение)
	\item В столбце хранятся данные одного типа
	\item Отсутствуют массивы и списки в любом виде
\end{itemize}

3.2 \textbf{Вторая нормальная форма}

Чтобы база данных находилась во второй нормальной форме (2NF), необходимо чтобы ее таблицы удовлетворяли следующим требованиям:

\begin{itemize}
	\item Таблица должна находиться в 1НФ
	\item Таблица должна иметь первичный ключ
	\item Все неключевые столбцы таблицы должны зависеть от полного ключа (в случае если он составной)
\end{itemize}

Если ключ составной, т.е. состоит из нескольких столбцов, то все остальные неключевые столбцы должны зависеть от всего ключа, т.е. от всех столбцов в этом ключе. Если какой-то атрибут (столбец) зависит только от одного столбца в ключе, значит, база данных не находится во второй нормальной форме.

3.3 \textbf{Третья нормальная форма}

Таблица должна находиться в 2НФ. Требование третьей нормальной формы (3NF) заключается в том, чтобы в таблицах отсутствовала транзитивная зависимость.

Транзитивная зависимость – это когда неключевые столбцы зависят от значений других неключевых столбцов.

\textbf{Нормальная форма Бойса-Кодда (НФБК)} (частная форма третьей нормальной формы)

\begin{itemize}
	\item Таблица должна находиться в 3НФ. 
	\item Ключевые атрибуты составного ключа не должны зависеть от неключевых атрибутов.
\end{itemize}

\color{olive} !!Есть вероятность, что дальше не нужно, так как на 2 семинаре модель бд мы проверяли до этого места (записи 1-ого семинара нет, так что проверить не могу)

В по этой теме полезно почитать про нотацию "Воронья лапка" и ER диаграммы
\color{black} 

3.4 \textbf{Четвертая нормальная форма}
Таблица должна находиться в НФБК.
Требование четвертой нормальной формы (4NF) заключается в том, чтобы в таблицах отсутствовали нетривиальные многозначные зависимости.

В таблицах многозначная зависимость выглядит следующим образом.

Начнем с того, что таблица должна иметь как минимум три столбца, допустим A, B и C, при этом B и C между собой никак не связаны и не зависят друг от друга, но по отдельности зависят от A, и для каждого значения A есть множество значений B, а также множество значений C.

В данном случае многозначная зависимость обозначается вот так:

A —> B

A —> C

Если подобная многозначная зависимость есть в таблице, то она не соответствует четвертой нормальной форме.

3.5 \textbf{Пятая нормальная форма}

Отношения находятся в 5НФ, если оно находится в 4НФ и отсутствуют сложные зависимые соединения между атрибутами.

Если «Атрибут1» зависит от «Атрибута2», а «Атрибут2» в свою очередь зависит от «Атрибута3», а «Атрибут3» зависит от «Атрибута1», то все три атрибута обязательно входят в один кортеж.

Это очень жесткое требование, которое можно выполнить лишь при дополнительных условиях. На практике трудно найти пример реализации этого требования в чистом виде.

\textit{Пояснение:} Например, некоторая таблица содержит три атрибута «Поставщик», «Товар» и «Покупатель». Покупатель1 приобретает несколько Товаров у Поставщика1. Покупатель1 приобрел новый Товар у Поставщика2. Тогда в силу изложенного выше требования Поставщик1 обязан поставлять Покупателю1 тот же самый новый Товар, а Поставщик2 должен поставлять Покупателю1, кроме нового Товара, всю номенклатуру Товаров Поставщика1. Этого на практике не бывает. Покупатель свободен в своем выборе товаров. Поэтому для устранения отмеченного затруднения все три атрибута разносят по разным отношениям (таблицам). После выделения трех новых отношений (Поставщик, Товар и Покупатель) необходимо помнить, что при извлечении информации (например, о покупателях и товарах) необходимо в запросе соединить все три отношения. Любая комбинация соединения двух отношений из трех неминуемо приведет к извлечению неверной (некорректной) информации. Некоторые СУБД снабжены специальными механизмами, устраняющими извлечение недостоверной информации. Тем не менее, следует придерживаться общей рекомендации: структуру базы данных строить таким образом, чтобы избежать применения 4НФ и 5НФ.



\textbf{Доменно-ключевая нормальная форма}

Ограничение домена – это ограничение, предписывающее использование для определенного атрибута значений только из некоторого заданного домена (набора значений).

Ограничение ключа – это ограничение, утверждающее, что некоторый атрибут или комбинация атрибутов представляет собой потенциальный ключ.

Таким образом, требование доменно-ключевой нормальной формы заключается в том, чтобы каждое наложенное ограничение на таблицу являлось логическим следствием ограничений доменов и ограничений ключей, которые накладываются на данную таблицу.

Таблица, находящаяся в доменно-ключевой нормальной форме, обязательно находится в 5NF. Однако, стоит отметить, что не всегда возможно привести таблицу к доменно-ключевой нормальной форме, более того, не всегда возможно получить ответ на вопрос о том, когда может быть выполнено такое приведение.

\textbf{Шестая нормальная форма}

Переменная отношения находится в шестой нормальной форме тогда и только тогда, когда она удовлетворяет всем нетривиальным зависимостям соединения. Из определения следует, что переменная находится в 6НФ тогда и только тогда, когда она неприводима, то есть не может быть подвергнута дальнейшей декомпозиции без потерь. Каждая переменная отношения, которая находится в 6НФ, также находится и в 5НФ.

Идея «декомпозиции до конца» выдвигалась до начала исследований в области хронологических данных, но не нашла поддержки.
