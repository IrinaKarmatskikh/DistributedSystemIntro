\newpage
\section{Билет 14. Свойства распределенных систем: конкурентность, отсутствие состояния, независимые сбои. Основные проблемы и аспекты проектирования: неоднородность, открытость, безопасность, масштабируемость, отказоустойчивость, конкурентность, прозрачность.}

\textcolor{olive}{Это вроде не часть билета, но для общего понимания о распределённых системах и более подробном описании свойств из этого билета:}

\textcolor{blue}{\href{https://novator.io/innovatsii/mir-raspredelyonnyh-sistem-ego-zakony-i-magiya-konsensusa-nakamoto}{Не очень строго про сами распределённые системы и их свойства}.}

\textcolor{blue}{\href{https://books.ifmo.ru/file/pdf/1551.pdf}{Более научно, можно прочитать часть 1.1-1.4}.}

\textbf{ Свойства распределённой системы:}
\begin{enumerate}
\item \textbf{Конкурентность}

Процессы в системе работают параллельно, т.е. одновременно происходит несколько событий. Другими словами, компьютеры сети выполняют свои задачи одновременно, но независимо друг от друга. Следовательно, необходима их координация.

Для того, чтобы распределённая система работала, необходимо иметь способ определения порядка, в каком происходят события. Однако, когда несколько компьютеров работают одновременно, зачастую невозможно сказать, какое из событий произошло раньше, а какое позже, поскольку компьютеры пространственно разнесены. Другими словами, нет единых глобальных часов, которые определяют последовательность событий, происходящих на компьютерах сети.

\textit{Как пример можно привести \textcolor{blue}{\href{https://ru.wikipedia.org/wiki/Задача_об_обедающих_философах}{ задачу об обедающих философах.}} \textcolor{olive}{(Постановка задачи в виккипедии немного отличается от формулировки Афонина, но суть та же)}}

\item \textbf{ Отсутствие состояния}

\textcolor{gray}{(Сорян, но я так и не понял, что тут имеется в виду).}

\item \textbf{ Независимые сбои}

Характерной чертой распределённых систем, которая отличает их от единичных компьютеров, является устойчивость к частичным отказам, т.е. система продолжает функционировать после частичных отказов.

\textit{Как пример можно привести \textcolor{blue}{\href{https://ru.wikipedia.org/wiki/Задача_византийских_генералов}{задачу византийских генералов.}} или билет 23}
\end{enumerate}

Когда разрабатывается или проектируется распределённая система, возникают следующие трудности:
\begin{enumerate}
\item
\textbf{ Неоднородность} той среды, в которой, разворачивается система (сети, вычислительные устройства, операционные системы, языки программирования, реализации однотипного ПО). В распределённую систему могут входить совершенно разные компоненты.

К примеру, значительная часть усилий программирования направленна на то, чтобы создать видимость простоты (Объектно-ориентированное проектирование).
То есть неоднородные элементы стараются спрятать за унифицыроваанным интерфейсом.

\item
\textbf{ Открытость} - способность системы к расширению, открытые интерфейсы. 

Открытые приложения - те приложения, с которыми можно общаться по общеизвестному протоколу. Например в протоколе http есть общедоступная спецификация, поэтому кем и как реализован клиент и сервер не важно, а протокол общения с базой данных оракла не опубликован и для связи нужно клиентское ПО.

\item
\textbf{ Безопасность} - политика безопасности, сопоставление правил подсистем, отказы в обслуживании, бозопасность данных.

В различных организациях существуют разные правила по работе с данными, которые нужно как-то объединить. 
Отказ в обслуживании (DDos атаки) также считают одиной из угороз безопасности, система должна быть высокодоступной. 
А также многие вопросы связанные с безопасностью данных, например чтобы данные нельзя было заменить неконтролируемым образом, безопасность передачи, конфиденциальность и т.д.

\item
\textbf{ Масштабируемость} - стоимость оборудования, производительноть ПО, анализ узких мест.

В общем случае масштабируемость определяют, как способность вычислительной системы эффективно справляться с увеличением числа пользователей или поддерживаемых ресурсов без потери производительности и без увеличения административной нагрузки на ее управление. При этом систему называют масштабируемой, если она способна увеличивать свою производительность при добавлении новых аппаратных средств.  

\item
\textbf{ Отказоустойчивость} - обнаружение и устранение ошибок, управление сбоями.

В распределённоё системе всё может пойти не так.
Многие сбои хочется устранять независимо для конечного пользователя.

\item
\textbf{ Конкурентность} - непротиворечивость данных.

Совместная работа пользователей тоже является проблемой.
Например заказ авиабилетов. Есть единая система бронирования, а точек, где можно продать билет много.

\item
\textbf{ Прозрачность}

\textcolor{olive}{Очень похоже на то, как об этом рассказывал Афонин -}\textcolor{blue}{\url{https://pnojournal.files.wordpress.com/2014/07/pdf_140605.pdf}}
\begin{itemize}
\item
\textit{ Прозрачность доступа.} (Локальные и удаленные)

Прозрачность в данном случае заключается в обеспечении сокрытия различий доступа и предоставлении данных.

\item
\textit{ Прозрачность местоположения. }

В распределенных системах прозрачность местоположения заключается в том, что пользователь не должен знать, где расположены необходимые ему ресурсы. 
Файлы могут перемещаться на различные узлы распределённой системы, но при этом, пользователь не должен замечать эти перемещения.

\item
\textit{ Прозрачность совместной работы.}

Различные пользователи распределенных систем должны иметь возможность параллельного доступа к общим данным. При этом необходимо обеспечить параллельное совместное использование ресурсами системы, а соответственно, обеспечить сокрытие факта совместного использования ресурсов.

\item
\textit{ Прозрачность репликации. }

В целях обеспечения сохранности данных, особенно на распределенных файловых системах, необходимо обеспечить репликацию данных. Пользователю не должно быть известно, что репликация данных существует.

\item
\textit{ Прозрачность восстановления после сбоев.}

Если что-то пошло не так, возможность востановления.

\item
\textit{ Прозрачность масштабировния.}

Масштабируемость распределённой системы является одной из важнейших характеристик распределенных систем и также не должа быть заметна конечному пользователю. До недавнего времени основной подход, позволяющий значительно увеличить мощность кластеров, заключался в наращивании различных ресурсов системы, к примеру, оперативной памяти, количества и объема жестких дисков. 
\end{itemize}

\end{enumerate}
