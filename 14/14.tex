\newpage
\section{Билет 14. Свойства распределенных систем: конкурентность, отсутствие состояния, независимые сбои. Основные проблемы и аспекты проектирования: неоднородность, открытость, безопасность, масштабируемость, отказоустойчивость, конкурентность, прозрачность.}

Распределенная  система (одно из возможных определений) – это набор  независимых  компьютеров,  не имеющих общей совместно используемой памяти и общего единого времени(таймера) и взаимодействующих через коммуникационную сеть  посредством  передачи  сообщений,  где  каждый  компьютер использует  свою  собственную  оперативную  память  и  на  котором выполняется  отдельный  экземпляр  своей  операционной  системы. Однако  эти  операционные  системы  функционируют  совместно, предоставляя свои службы друг другу для решения общей задачи.

\textbf{ Свойства распределённой системы:}
\begin{enumerate}
\item \textbf{Конкурентность}

Процессы в системе работают параллельно, т.е. одновременно происходит несколько событий. Другими словами, компьютеры сети выполняют свои задачи одновременно, но независимо друг от друга. Следовательно, необходима их координация.

Для того, чтобы распределённая система работала, необходимо иметь способ определения порядка, в каком происходят события. Однако, когда несколько компьютеров работают одновременно, зачастую невозможно сказать, какое из событий произошло раньше, а какое позже, поскольку компьютеры пространственно разнесены. Другими словами, нет единых глобальных часов, которые определяют последовательность событий, происходящих на компьютерах сети.

\textit{Показательный пример этого свойства - \textcolor{blue}{\href{https://clck.ru/puqyK}{задача об обедающих философах.}}}

\item \textbf{ Отсутствие состояния}

Поскольку в распределённых системах нет понятия совместного используемой памяти, часто трудно определить текущее состояние системы. Определение глобального состояния распределённой системы можно выполнить путём синхронизации всех процессов так, чтобы каждый из них определил и сохранил своё локально состояние вместе с сообщениями, которые передаются в этот момент. Сама по себе синхронизация может быть выполнена без остановки процессов и записи их состояния. Вместо этого в ходе работы распределённой системы с неё можно сделать мгновенный слепок - распределённой снимок состояния.

\item \textbf{ Независимые сбои}

Характерной чертой распределённых систем, которая отличает их от единичных компьютеров, является устойчивость к частичным отказам, т.е. система продолжает функционировать после частичных отказов.

\textit{Показательный пример этого свойства - \textcolor{blue}{\href{https://clck.ru/JVS7z}{задача византийских генералов}} (также описана в билете 23)}
\end{enumerate}

\noindent При разработке или проектировании распределённой системы, обычно возникают следующие проблемы:
\begin{enumerate}
\setlength\itemsep{0.32em}
\item
\textbf{ Неоднородность} той среды, в которой, разворачивается система (вычислительные устройства, операционные системы, языки программирования, реализации однотипного ПО). 

В распределённую систему могут входить совершенно разные компоненты. Что вызывает трудности при из взаимодействии.
К примеру, значительная часть усилий программирования направленна на то, чтобы создать видимость простоты (Объектно-ориентированное проектирование).
То есть неоднородные элементы стараются спрятать за унифицированным интерфейсом.

\item
\textbf{ Открытость} - способность системы к расширению, открытые интерфейсы. 

Открытые приложения - те приложения, с которыми можно общаться по общеизвестному протоколу. Например в протоколе http есть общедоступная спецификация, поэтому, кем и как реализован клиент и сервер, неважно, а протокол общения с базой данных оракла не опубликован и для связи с ней нужно клиентское ПО.

\item
\textbf{ Безопасность} - политика безопасности, сопоставление правил подсистем, отказы в обслуживании, безопасность данных.

В различных организациях существуют разные правила по работе с данными, которые нужно как-то объединить. 
Отказ в обслуживании (DDos атаки) также считают одной из угроз безопасности, система должна быть высокодоступной. 
А также необходимо учесть многие вопросы, связанные с безопасностью данных, например, чтобы данные нельзя было заменить неконтролируемым образом, безопасность передачи, конфиденциальность и т.д.

\item
\textbf{ Масштабируемость} - стоимость оборудования, производительность ПО, анализ узких мест.

В общем случае масштабируемость определяют, как способность вычислительной системы эффективно справляться с увеличением числа пользователей или поддерживаемых ресурсов без потери производительности и без увеличения административной нагрузки на ее управление. При этом систему называют масштабируемой, если она способна увеличивать свою производительность при добавлении новых аппаратных средств.  

\item
\textbf{ Отказоустойчивость} - обнаружение и устранение ошибок, управление сбоями.

В распределённой системе всё может пойти не так.
Многие сбои хочется устранять независимо для конечного пользователя.

\item
\textbf{ Конкурентность} - непротиворечивость данных.

Совместная работа пользователей тоже является проблемой.
Например заказ авиабилетов. Есть единая система бронирования, а точек, где можно продать билет много.

\item
\textbf{ Прозрачность}

\begin{itemize}
\item
\textit{ Прозрачность доступа.} (Локальные и удаленные)

Прозрачность в данном случае заключается в обеспечении сокрытия различий доступа и предоставлении данных.

\item
\textit{ Прозрачность местоположения. }

В распределенных системах прозрачность местоположения заключается в том, что пользователь не должен знать, где расположены необходимые ему ресурсы. 
Файлы могут перемещаться на различные узлы распределённой системы, но при этом, пользователь не должен замечать эти перемещения.

\item
\textit{ Прозрачность совместной работы.}

Различные пользователи распределенных систем должны иметь возможность параллельного доступа к общим данным. При этом необходимо обеспечить параллельное совместное использование ресурсами системы, а соответственно, обеспечить сокрытие факта совместного использования ресурсов.

\item
\textit{ Прозрачность репликации. }

В целях обеспечения сохранности данных, особенно на распределенных файловых системах, необходимо обеспечить репликацию данных. Пользователю не должно быть известно, что репликация данных существует.

\item
\textit{ Прозрачность восстановления после сбоев.}

Если что-то пошло не так, возможность восстановления.

\item
\textit{ Прозрачность масштабировния.}

Масштабируемость распределённой системы также не должна быть заметна конечному пользователю. До недавнего времени основной подход, позволяющий значительно увеличить мощность кластеров, заключался в наращивании различных ресурсов системы, к примеру, оперативной памяти, количества и объема жестких дисков. 
\end{itemize}

\end{enumerate}
