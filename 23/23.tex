\newpage
\section{Билет 23. Согласие в распределенной системе. Задача византийских генералов: распространение значения (agreement), согласование решения (consensus), согласование вектора (interactive consistency). Невозможность решения при трёх процессах и одном сбое. Алгоритм Лэмпорта для "устных" сообщений. Пример распределенных транзакций.}

\textbf{Согласие в распределённой системе.} 

Постановка задачи: \\
Задано N процессов, у каждого есть некие данные — предложение (proposal), они должны выполнить некоторый распределённый алгоритм и прийти к решению (decision).

Требования: \\
Основные:
\begin{itemize}
\item \textbf{Согласие (agreement)}: все не отказавшие (не упавшие навсегда) процессы должны завершиться с решением (decide) и все эти решения должны совпадать.
\item \textbf{Нетривиальность (non-triviality)}: должны быть варианты исполнения, приводящие к разным решениям (возможно, просто с разными исходными предложениями или разным исходным состоянием процессов).
\end{itemize}
Дополнительные: 
\begin{itemize}
\item \textbf{завершение (termination)}: протокол должен завершиться за конечное время.
\end{itemize}

\textbf{Задача византийских генералов} 

Формулировка: 

Византия. Ночь перед великим сражением с противником. Византийская армия состоит из N легионов, каждым из которых командует свой генерал. Также у армии есть главнокомандующий, которому подчиняются генералы.В то же самое время империя находится в упадке, и любой из генералов и даже главнокомандующий могут быть предателями Византии, заинтересованными в её поражении.Ночью каждый из генералов получает от главнокомандующего приказ, как надлежит поступить в 10 часов утра (время одинаковое для всех и известно заранее). Варианты приказа: «атаковать противника» или «отступать».

Возможные исходы сражения:
\begin{enumerate}
\item Если все верные генералы атакуют -- Византия уничтожит противника (благоприятный исход).
\item Если все верные генералы отступят -- Византия сохранит свою армию (промежуточный исход).
\item Если некоторые верные генералы атакуют, а некоторые отступят -- противник со временем по частям уничтожит всю армию Византии (неблагоприятный исход).
\end{enumerate}
   
Также следует учитывать, что если главнокомандующий — предатель, то он может дать разным генералам противоположные приказы, чтобы обеспечить уничтожение армии. Следовательно, генералам надо учитывать такую возможность и не допускать несогласованных действий. Если же каждый генерал будет действовать полностью независимо от других (например, сделает случайный выбор), то вероятность благоприятного исхода весьма низка.Поэтому генералы нуждаются в обмене информацией между собой, чтобы прийти к единому решению. 