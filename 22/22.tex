\newpage
\section {Билет 22. Взаимное исключение. Решение на основе сервера. Алгоритм Дейкстры с общими регистрами. Алгоритм Петерсона для двух процессов. Алгоритм на основе голосования (Maekawa).}
\textbf{Взаимное исключение}.

Постановка задачи: \\
Задано N процессов, конкурирующих за \textbf{общий} ресурс. Каждый процесс циклически выполняет код, принадлежащий \textbf{критической секции} (CS), и не принадлежащий ей (т.е. иногда процесс заходит в критическую секцию и потом из неё выходит).
Задача стоит в разработке алгоритма, который гарантирует, что в каждый момент времени в критической секции находится не более одного процесса.

Требования: \\
Основные:
\begin{itemize}
\item \textbf{Safety}: Не более одного процесса находится в критической секции
\item \textbf{Liveness}: Если один или несколько процессов хотят зайти в CS, то рано или поздно по крайней мере один из них зайдет в критическую секцию (при условии, что время нахождения процессов в CS ограничено)
\end{itemize}
Дополнительные:
\begin{itemize}
\item \textbf{Отсутствие голодания}: Если процесс p хочет войти в критическую секцию, то рано или поздно он войдет в критическую секцию \footnote{\textit{прим. авт.:} из Liveness это условие не следует} 
\item \textbf{Справедливость}: Если процесс p захотел зайти в CS раньше процесса q, то он зайдет в неё раньше
\end{itemize}

Задача ставится не только для систем с распределенной памятью, но и для систем с общей памятью: процессы могут взаимодействовать через общие переменные; атомарны только операции чтения или записи.

\begin{algorithm}
\caption{Алгоритм Деккера}
Решение задачи взаимного исключения для двух процессов: 
\label{algDekker}
\begin{algorithmic}
\Ensure \\$status[2]$ \Comment{массив из двух чисел. Для чтения каждому из процессов. Для записи у $i$-ого процесса есть доступ только к $i$ ячейке.}\\ 
$turn$ \Comment{переменная, в которую оба процесса могут записывать и читать данные}\\ 
\While{$status[other] == competing$} 
  \If{$turn == other$} 
    \State $status[i] = out$
    \State $wait until (turn == i)$
    \State $status[i] = competing$
  \EndIf
\EndWhile

\State \textbf{CS}
\State $turn = other$
\State $status[i] = out$
\end{algorithmic}
\end{algorithm}


\begin{algorithm}
\caption{Алгоритм Дейкстры}
Решение задачи взаимного исключения для N процессов: 
\label{algDijkstra}
\begin{algorithmic}
\Ensure \\$status[i] \in \{competing, out, cs\}$ \\
$turn \in \{1,...N\}$ 

\Do
  \While{$turn \neq i$} 

    \If{$status[turn] == out$} 
      \State $turn := i$
    \EndIf
  \EndWhile
  \State $status[i] = cs$
\doWhile{$\exists other:\ status[other] = cs$}
\State \textbf{CS}
\State $status[i] = out$
\end{algorithmic}
\end{algorithm}

\begin{algorithm}
\caption{Алгоритм Петерсона}
Решение задачи взаимного исключения для 2 процессов: 
\label{algPeterson}
\begin{algorithmic}
\State Перед тем как начать исполнение критической секции кода, процесс должен вызвать процедуру LOCK() со своим номером в качестве параметра. Она должна организовать ожидание процессом своей очереди входа в критическую секцию. После исполнения критической секции и выхода из неё процесс вызывает другую процедуру UNLOCK(), после чего уже другой процесс сможет войти в критическую область. 
\Ensure \\$want[2]$ \Comment{массив из двух чисел}\\ 
$victim$ \Comment{переменная, в которую оба процесса могут записывать и читать данные}\\ 
\Procedure{lock:}{}
  \State $want[i] = true$  
  \State $victim = i$
  \While{$want[1-i] == true$ \textbf{and} $victim == i$}
    \State pass
  \EndWhile
\EndProcedure

\\
\Procedure{unlock:}{}
  \State $want[i] = false$
\EndProcedure
\end{algorithmic}
\end{algorithm}


\nameref{algDijkstra} не удовлетворяет требованию справедливости (нужный процесс зайдет в критическую секцию, а остальные могут зациклиться в алгоритме)
