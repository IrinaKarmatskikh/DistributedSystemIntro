\newpage
\section{Билет 16. Архитектуры информационных систем: "монолитная", клиент-серверная, многоуровневая. Одноранговые системы (peer-to-peer).}

\textbf{Компоненты архитектуры распределённой системы:}
\begin{enumerate}
\item \textit{Физическое оборудование и сеть}

Приложение запускается на компьюторе определённого производителя, определённого типаа памяти и тд.

\item \textit{Операционная система.}

Устанавливается на физическое оборудование. Одна из задач - предоставить единый интерфейс представления оборудования. То естьт если написать программу на линукс, то она будет работать на любом линуксе. 

\item \textit{Промежуточное ПО (Middleware).}

Сокрытие неоднородности и предоставление базовых функций приложениям (удалённый вызов функции, примитивы передачи сообщений, многоадресная передача и т.п.).
Набор служб которые полезны многим приложениям.

\item \textit{Приложения и сервисы.}

Непосредственно разработанное приложение.
\end{enumerate}

Цель каждого уровня - максимально скрыть сложность уровня, который был выше.\\
\\
Различные систтемные архитектуры:
\begin{enumerate}
\item \textbf{ Монолитное решение}

Всё работает на одном компьютере.

\textit{ Примечание: это не распределённая система.}

\item \textbf{ Клиент-серверная}

Базы данных обычно вынесена на отдельный сервер. Есть программа клиент и программа сервер, а также какой-то прототкол их общения, желательно открытый.

Базы данных различных разработчиков практически всегда отличаются, и взимозаменяемости (открытости) на уровне протоколов почти всегда нет, поэтому она достигаетя за счёт стандартизации запросов (SQL).

\item \textbf{ Клиент-серверная с посредниками (proxy-сервер)}

Есть промежуточные сервера, которые со стороны клиента выглядят так же, но такой сервер ничего не делает, а просто перенаправляет запросы. Иногда proxy-сервер может закэшировать какие-то данные или решать какие-то вопросы безопасности (например доступ к главному серверу только через proxy).

\item \textbf{ Многоуровневая}
Может быть двух типов:
\begin{itemize}
\item
Многосерверная - сервис предоставляется совокупностью вычислительных процессов (например репликация, которая реализованна в виде нескольких общающихся процессов, но для клиента выглядят одним). По большому счёту это клиент-сервер.

\item
Сервера приложений - может быть много серверов и один сервер может быть клиентом другого сервера.
Например подход MVC (Model-View-Controller): клиент <--> сервер <--> база данных.

MVC - схема разделения данных приложения и управляющей логики на три отдельных компонента: модель (хранение данных), представление и контроллер (обработка) — таким образом, что модификация каждого компонента может осуществляться независимо

\end{itemize}

\item \textbf{ Одноранговая}

Какой либо сервер не выделяется, все равны (например торрент).
\end{enumerate}



















